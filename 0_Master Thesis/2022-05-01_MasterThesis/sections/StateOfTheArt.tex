\chapter{State of the Art in Virtual Commissioning}
\section{Visualization in the PLC Development Environment}			% TwinCAT
	The \ac{HMI} offers a quick possibility for testing the control software. In most cases, a graphical user interface can be created directly in the development environment in which current values and outputs of the control software can be displayed and inputs can be set.\\
	Bei dieser Herangehesweise ahmt der menschliche Entwickler das Verhalten der realen Anlage nach und versucht damit Probleme in der Software zu identifizieren. \\
	Das Erstellen einer \ac{HMI} wird von vielen gängigen SPS-Herstellern wie zum Beispiel Siemens und Beckhoff unterstützt. 
	
\section{Visualization using an Physics Engine}		% Unity		
	Die nächste Ausbaustufe in der virtuellen Inbetriebnahme ist die Verwendung einer externen physics engine wie zum Beispiel Unity. Dabei können die originalen Geometriedaten importiert werden, was eine bessere Übersicht bietet. Vorallem mit vorgefertigten Libraries kann schnell ein Model aufgebaut werden. 
Nachteilig an dieser Variente ist einen nötige Kommunikation zwischen der PLC runtime und der physics engine. 

\section{Digital Twin in Simulation Software}		% Simulink
	Die Verwendung einer Simulations-Software wie Matlab/Simulink ist vorallem bei größeren Systemen oder Regelungen vorteilhaft. Vorallem eine graphische Oberfläche bieten eine gute Übersicht und verringern damit den nötigen Zeitaufwand in der Erstellung von Modellen oder ihrer Wartung.

\section{Hardware in the Loop (HIL)}


\section{Available Data Formats}
\subsection{Geometry Data}		% step, COLLADA (kin)
	Die Industrie und ihre Produkte sind sehr vielfältig und damit auch die Anforderungen an die CAD-Software. Je nach Anwendungsfall kann dadurch eine Software vorteilhafter sein als eine andere. Ein Hauptunterschied dabei ist \textit{direkte Modellierung} im Vergleich zur \textit{parametrisierten Modellierung}. 
	Bei der direkten Modellierung wird die Geometrie mithilfe von fixen Werten erzeugt. Daraus resultieren keine Abhängigkeiten zwischen den verschiedenen Elementen der Geometrie. Diese Art der CAD-Software wird vor allem im statischen Bereich, wie zum Beispiel im Stahlbau, eingesetzt.
	Im Gegenzug dazu wird bei der parametrisierten Modellierung die Geometrie mithilfe von Abhängigkeiten und Parametern erzeugt. Dies ist vor allem im dynamischen Bereich mit beweglichen Teilen von Vorteil. 
	Große Anbieter von CAD-Software sind dabei: \textit{Creo Elements}, \textit{Autodesk Inventor} oder \textit{Siemens NX}. 
	Bei dem Import von CAD-Modellen muss auf das Austauschformat geachtet werden, da nicht alle die Kinematisierung über Abhängigkeiten unterstützen. Ein Beispiel, wo die Kinematisierung beibehalten wird, wäre \ac{COLLADA} und ein Gegenbeispiel das \ac{STEP} oder auch \ac{IGES} Format.

\subsection{Behavior Models}	% FMI https://fmi-standard.org/
	Ähnlich der Geometriedaten ergibt sich eine Vielzahl von möglichen Dateiformate abhängig von der Disziplin und der verwendeten Simulationssoftware für die Beschreibung des Verhaltensmodells. Besonders im Bereich der Co-Simulation wird jedoch eine universelle Schnittstelle gefordert. Aus diesem Grund wurde bereits im Jahr 2010 das \ac{FMI} von \textit{Modelica Association} definiert. Die Grundlage von diesem Interface beruht auf C-Code, welcher universell eingesetzt werden kann. Aktuell liegt dieses Interface in der Version 2.0.3 (November 2022) vor und wird bereits von über 100 Tools unterstützt. Der Quellcode ist dabei Opensource und wird unter der \textit{2-Clause BSD} Lizenz vertrieben.  

\subsection{Code Snippets and Examples}		% PLCopen
	Im Bereich der Steuerungstechnik mithilfe von SPS ist vor allem die EN~61131 der Grundpfeiler für die verwendeten Programmiersprachen. Diese Norm beruht zu einem großen Teil auf der Organisation \textit{PLCoopen}, welche sich zum Ziel gesetzt hat die Effizienz bei der Erstellung von Steuerungssoftware zu steigern und dabei plattformunabhängig zu sein. Die großen SPS-Hersteller stützen sich dabei auf diese Norm und bieten Schnittstellen für die definierten Programmiersprachen. Damit kann der Aufwand beim Austausch der Software minimiert werden. 


\subsection{Documentation}		% PDF, MD, HTML
	Nur mit einer guten Dokumentation stellt eine ordnungsgemäße Verwendung eines Produktes sicher. Diese sollte leicht für den Menschen lesbar und verständlich sein. Hierzu bieten sich verschiedene Dateiformate an wie: \ac{PDF}, \ac{MD} oder \ac{HTML}. Jedes Format bietet andere Vorteile wie etwa eine einfache Handhabung, direkte Git-Integration oder auch eine hohe Kompatibilität über verschiedene Geräte. 

\subsection{Exchange Libraries}


%\section{Version Control}








