\section{Creation of Data Structure - Workflow of a Manufacturer}	
    % Allgemeine Infos
    In this section the data structure for the exchange is generated. This structure consists of the CAD data, the behaviour model and an exemplary control as described in \autoref{sec:DataStructure}. The export of the data is done from the view of the supplier or the transmitter. \\
    
    % CAD
    For exporting the CAD data, this example uses the native format of \textit{Inventor} for kinematized models. That is based on the assumption that the transmitter and receiver of the data structure use \textit{Inventor}. The structure of the assembly corresponds to the desired kinematization, taking into account required degrees of freedom. If the assembly is purely static and thus no kinematicization is required, the \textit{.step} format is used for exchange. This also simplifies the structure of the assembly, whereby logically linked components are bundled into individual assemblies. The export to the \textit{.step} format is done using the instruction \cite{InventorAnleitungExportStep} and to the native exchange format using the instruction \cite{InventorAnleitungExportPackAndGo}. \\
    
    % Verhaltensmodell
    As already mentioned, in this example the original models from \textit{Simulink} are used as an alternative for the integration into \textit{TwinCAT}. However, the behaviour model is exported as intended as a \textit{.fmu} file and included in the data structure. This is done according to the instructions \cite{MatlabFmuExport}, using a \textit{fixed-step} solver with the step size of \SI{10}{\milli\second}. The step size corresponds to the cycle time of the PLC. \\
    The interface of the models reproduces the signals of the real hardware and describes the given system. In addition, some purely virtual variables are used, which are useful for higher-level control. The level of detail can be arbitrarily precise, whereby the increasing complexity of the required calculations must be taken into account. 
    
    % PLC 
    The source code of the PLC is exported to the \textit{PLCopen} format for exchange. Instructions for this are provided by \cite{TwincatExportPlcopen}.

    
\subsection{Module \textit{Seperation}}
\subsubsection{CAD Data}
Das CAD-Model wird anhand der gewünschten Kinematisierung der Zylinder aufgebaut. Dabei stellen die beiden Zylinder jeweils eine eigene Baugruppe dar, um die Wartung und die Verwendung zu erleichtern. Wie schon erwähnt wird die Baugruppe aufgrund der gewünschten Kinematisierung im nativen Format von \textit{Inventor} exportiert und in einer \textit{.zip}-Datei gebündelt. 

    
\subsubsection{Behaviour Model}
% Verhaltensmodell
    The behaviour model in this case is created in \textit{Simulink} and takes into account the influence of gravity on the containers in the incoming storage. Also taken into account is the influence of the actuators on the selected position of the containers in the storage and the resulting signal from the sensors. The entire model with the inputs and outputs is shown in \autoref{fig:ModuleSeperationBehaviourModel}.  \\
    
% Interface
    For this module, the interface consists of digital outputs for moving the two actuators and digital inputs for each of the three proximity sensors. Furthermore, additional signals are needed for the creation and deletion of the containers. These signals are purely virtual and are thus not connected to the hardware, but are required only in the logic of the station.\\

% Beschreibung Model
	

% Export zu FMI

\subsubsection{PLC Code}
    For the exemplary control of the \textit{separation} a \textit{function block} is created in \textit{TwinCAT}. By structuring the control as a \textit{function block}, an instance of the \textit{separation} can easily be created and used in a higher-level controller. This code basically consists of a state machine in which the two pistons are controlled depending on the current state and the sensor inputs. The resulting \textit{.xml} file is listed in \autoref{lst:FbSeperation}. 

\subsubsection{Resulting File}
    The generated information of the CAD model, behaviour and control is now bundled in the proposed data structure from \autoref{sec:DataStructure}. The resulting structure for this example is shown in \autoref{fig:ExampleVereinzelungFile}. 
	

\subsection{Module \textit{Conveyor Belt}}
\subsubsection{CAD Data}
    In this module the CAD model of the conveyor provided by the manufacturer as a \textit{.step} file. This \textit{.step} file is included in the assembly after it is imported and converted to a native format of \textit{Inventor}. The three sliders of the stoppers form again a sub-assembly to simplify the kinematization. The conveyor itself does not need any additional kinematization. The remaining components such as the conveyor itself, the attachment, and the sensors are bundled into a second assembly to avoid any obstacles to the kinematics. The three dosing units are also represented as a own sub-assembly. \\
  
   Since the kinematics of the pistons are supposed to be exported in this example as well, the native CAD format is kept. Alternatively, a pure exchange of the geometry with an \textit{.step} file can be done.

\subsubsection{Behaviour Model}
    The behaviour model in this module consists of the DC motor of the conveyor belt, a generic drive for the dosing units and a logical combination of the actuators and sensors. The modeling of the DC motor in this case is similar to this example from the Matlab documentation \cite{SimulinkExampleDcMotorControl}, using the characteristics from the data sheet of the used motor \cite{DataSheetDCMotor}. \\
    The stepper motors of the dosing units are not modeled as detailed as the DC motor and therefore use a generic description, which also results in a reduction of the required computing power. \\
    The final model is shown in \autoref{fig:ExampleFillingModelDrive}.  \\
    


\subsubsection{PLC Code}
   The PLC code in this module consists of two function blocks: one block for the conveyor belt and one block for the dosing units. The linking of these blocks remains the customer's task and is not supplied by the manufacturer. The two function blocks are listed in \textit{PLCopen} format in \autoref{lst:FbConveyorBelt} and \autoref{lst:FbSDosingUnit}. 
    

\subsubsection{Resulting File}
    The collected information is now merged into the proposed data structure from \autoref{sec:DataStructure}. This data structure is shown in \autoref{fig:ExampleFillingFile} and consists of the data of the CAD assembly, the behaviour model and the documentation consisting of the data sheet of the DC motor with the encoder. 
	

\subsection{Module \textit{Cartesian Gripper}}
    \subsubsection{CAD Data}
    Similar to the previous modules, the structure of the CAD data reflects the desired kinematization. For this reason, the \textit{cartesian gripper} consists of three assemblies for the respective axes and one for the remaining components. The degrees of freedom of the three axes are restricted to the desired direction of motion. Since the CAD data along with the kinematization will be sent to the customer, the native \textit{Inventor} format is used again.
    
    \subsubsection{Behaviour Model}
    The behaviour model of the \textit{cartesian gripper} maps the three axes of motion and the gripper itself. In the process, a stepper motor is merged with the terminal of the PLC into one model and described generically. As a result of the generic model, the level of detail is reduced and so is the complexity of the calculation. Furthermore, the generic model is independent of the selected motor type.
    	
	
	\subsubsection{PLC Code}
	In this module, a function block is provided for controlling a motor axis. This function block controls only one axis and must be instantiated later for each motor. This block thereby independently performs the homing to the zero point allowing an accurate movement to a target point with a desired speed. Again, the control is exported to the \textit{PLCopen} format and is listed in \autoref{lst:FbManipulatorAxis}.
	
	\subsubsection{Resulting File}
	Der letzte Schritt besteht nun aus dem Zusammenführen der Daten in die vorgeschlagene Datenstruktur aus \autoref{sec:DataStructure}. Diese Struktur ist in \autoref{fig:ExampleCarthesianGripperFile} dargestellt und besteht aus der CAD Baugruppe, dem Verhaltensmodell und dem Funktionsblock für die Steuerung. 

    
\subsection{Module \textit{Load Cell}}
    \subsubsection{CAD Data}
    Since no kinematization is required for this module, no special steps are necessary in the construction of the CAD assembly. In addition, data exchange is simplified since a \textit{.step} file can be used without losing any relevant data.
    
    \subsubsection{Behaviour Model}
    The generation of the behaviour model of this module is similarly simple, because the serial communication itself is not modeled but only the message as \textbf{string}-type. With this behaviour model a message can be created with a defined prefix and suffix and a desired measured value, which later has to be interpreted by the PLC. This model is represented in \autoref{fig:ModuleLoadCellBehaviourModel}.

	\subsubsection{PLC Code}
	Here the manufacturer provides only the hardware and the documentation of the communication, but no code for the PLC. 
	
	\subsubsection{Resulting File}
	Auch in diesem Modul besteht der letzte Schritt aus dem Zusammenführen der Daten in die Datenstruktur. Die resultierende Struktur ist in \autoref{fig:ExampleLoadCellFile} dargestellt und besteht aus der CAD Baugruppe, dem Verhaltensmodell und der Dokumentation. 

    
\subsection{Module \textit{Thermal Processing}}
    \subsubsection{CAD Data}
    This module is also based on static components and therefore has no kinematization. This simplifies on the one hand the setup in the CAD software and on the other hand the export of the data in the \textit{.step} format.
    
    \subsubsection{Behaviour Model}
    However, the behaviour model is slightly more complex with the modeling of the temperature curve. Taking into account the room temperature and the maximum reachable temperature of the cartridge heaters, the temperature of the containers is described with a transfer function. Finally, this temperature has to be converted into the corresponding signals of the sensors. The entire model in \textit{Simulink} is shown in \autoref{fig:ModuleThermalProcessingBehaviourModel} for this purpose. 
   
    
    \subsubsection{PLC Code}
    With this module the PLC code is again provided in the form of a \textit{function block}. This block is listed in \autoref{lst:FbManipulatorAxis} and enables the initialization and operation of the module. In the process, the containers are heated to a defined temperature and then held for a certain time. 
    
    
    \subsubsection{Resulting File}
    Abschließend werden die gesammelten Daten in der Datenstruktur zusammengeführt. Sie ist in \autoref{fig:ExampleThermalProcessingFile} dargestellt und besteht aus den CAD-Daten, dem Verhaltensmodell und dem SPS-Code.
	