
% Sprache festlegen
%		\ifthenelse{\boolean{english}} {\usepackage[ngerman,english]{babel} } 	 {\usepackage[english,ngerman]{babel}}
		\ifthenelse{\boolean{english}} {\usepackage[english]{babel} } 	 {\usepackage[ngerman]{babel}}


% Captions 	--- von Geiger D.
	% Text ausschreiben
%		\addto\captionsenglish{\renewcommand{\figurename}{Fig.}}
%		\addto\captionsenglish{\renewcommand{\tablename}{Tab.}}
	% Schriftart
		\usepackage{helvet}
		\renewcommand{\familydefault}{phv}
		\usepackage[font={small},format=hang]{caption}
		\usepackage[font={small},format=hang]{subcaption}


% Einheiten und Zahlen
	\usepackage{units}
	\usepackage{siunitx}
%	\sisetup{exponent-product=\cdot,output-product=\cdot,per-mode=symbol-or-fraction,decimalsymbol=comma,detect-weight=true, round-mode=off, round-precision=2}
	\sisetup{exponent-product=\cdot,output-product=\cdot, round-precision=2}
	\DeclareSIUnit \voltampere {VA} 
	\DeclareSIUnit \var {var}  
	\DeclareSIUnit \baud {baud} 
	\DeclareSIUnit \dbm {dBm} 



% Schaltungen Zeichnen
%	\usepackage{circuitikz}
%	\ctikzset{bipoles/length=1cm}
%	\ctikzset{bipoles/nos/height=.8, bipoles/nos/width=.8,}

% Nummerierung der Bilder pro Chapter	--- von Geiger D.
	\usepackage{chngcntr}
	\counterwithin{figure}{chapter}
	\counterwithin{table}{chapter}
	\counterwithin{equation}{chapter}



% layout
	\usepackage{fancyhdr}
	\setlength{\headheight}{14pt}
	
	\usepackage{wallpaper}
	\usepackage{float}
	\usepackage[colorlinks=true,urlcolor=black,linkcolor=black]{hyperref}  
	\hypersetup{ pdftitle= {xxx}, pdfauthor={xxx}, pdfsubject = {xxx}, citecolor=black	}
	
	\usepackage{paralist}	%% Kompakte Aufzählung
	\parindent 0pt	% Einzug bei neuem Abschnitt auf 0
	\usepackage[numbers]{natbib}	%% Zitierung

	\usepackage{framed} %% für die linken Balken bei den Beispielen

% graphics
	\usepackage{epstopdf}
	\usepackage{pst-all}
	\usepackage{float}
	%\usepackage[dvips,clip]{graphicx}
	\usepackage{graphicx}
	\usepackage{exscale,relsize}
	\usepackage{psfrag}
	\usepackage[percent]{overpic}
	
	\newcommand\blankpage{
	\null
	\thispagestyle{empty}
	\addtocounter{page}{-1}
	\newpage}

%\usepackage[duplicate]{chapterl.bib}
%\usepackage[refsection=part,backend=biber]{biblatex}
\usepackage{minitoc}

% equations
	\usepackage{amsmath}
	\usepackage{amssymb}
	\usepackage{amsthm}
	\usepackage{mathrsfs}
	\usepackage{amsxtra}
	\usepackage{xfrac}
	\usepackage{bbding}
	\usepackage{amsbsy}
	
	
% including code
	\usepackage{verbatim}
	\usepackage{moreverb}
	\usepackage{url}


% Aufzählungszeichen
	\renewcommand{\labelitemi}{-}


% Kopfzeile
	\pagestyle{fancy}
	\renewcommand{\subsectionmark}[1]{}
	\renewcommand{\sectionmark}[1]{\markright{#1}} % clears the section number in the header
	\lhead{\nouppercase{\leftmark}}
	\rhead{\thepage}
	\cfoot{}
	\renewcommand{\headrulewidth}{0.4pt}

% Fußzeile  
%	\lfoot{{\it\footnotesize \lecture}}
%	\rfoot{{\it\footnotesize \semester}}
	\renewcommand{\footrulewidth}{0.0pt}



% Für Programmcode im Text
	\usepackage{listings} 
	\usepackage{listings}
\usepackage{color}

\definecolor{mygreen}{rgb}{0,0.6,0}
\definecolor{mygray}{rgb}{0.5,0.5,0.5}
\definecolor{mymauve}{rgb}{1,0,0.2}

	\lstloadlanguages{C}
	\lstset{ 
		backgroundcolor=\color{white},   % choose the background color; you must add \usepackage{color} or \usepackage{xcolor}; should come as last argument
		basicstyle=\footnotesize,        % the size of the fonts that are used for the code
		breakatwhitespace=false,         % sets if automatic breaks should only happen at whitespace
		breaklines=true,                 % sets automatic line breaking
		captionpos=t,                    % sets the caption-position to bottom
		commentstyle=\color{mygreen},    % comment style
		deletekeywords={...},            % if you want to delete keywords from the given language
		escapeinside={\%*}{*)},          % if you want to add LaTeX within your code
		extendedchars=true,              % lets you use non-ASCII characters; for 8-bits encodings only, does not work with UTF-8
		firstnumber=1,                % start line enumeration with line 1
		frame=single,	                   % adds a frame around the code
		keepspaces=true,                 % keeps spaces in text, useful for keeping indentation of code (possibly needs columns=flexible)
		keywordstyle=\color{black},       % keyword style
		language=C,                 % the language of the code
		morekeywords={*,...},            % if you want to add more keywords to the set
		numbers=left,                    % where to put the line-numbers; possible values are (none, left, right)
		numbersep=5pt,                   % how far the line-numbers are from the code
		numberstyle=\tiny\color{mygray}, % the style that is used for the line-numbers
		rulecolor=\color{black},         % if not set, the frame-color may be changed on line-breaks within not-black text (e.g. comments (green here))
		showspaces=false,                % show spaces everywhere adding particular underscores; it overrides 'showstringspaces'
		showstringspaces=false,          % underline spaces within strings only
		showtabs=false,                  % show tabs within strings adding particular underscores
		stepnumber=2,                    % the step between two line-numbers. If it's 1, each line will be numbered
		stringstyle=\color{mymauve},     % string literal style
		tabsize=2,	                   % sets default tabsize to 2 spaces
		title=\lstname                   % show the filename of files included with \lstinputlisting; also try caption instead of title
	}

%	\ifthenelse{\boolean{english}}			 {\renewcommand{\lstlistlistingname}{List of Code}} 	 {\renewcommand{\lstlistlistingname}{Quellcodeverzeichnis}}		%%Überschrift von Verzeichnis




% mcode package for MATLAB code highlighting
%\usepackage{sub/mcode}
%\lstloadlanguages{matlab}
%
%% Setup the listings settings
%\lstset{
%	frame=leftline,
%	rulecolor=\color{gray},
%	escapeinside={(*@}{@*)},
%	language=C,
%	aboveskip=3mm,
%	belowskip=3mm,
%	showstringspaces=false,
%	columns=flexible,
%	basicstyle={\fontfamily{pxtt}\selectfont\footnotesize},
%	numbers=left,
%	stepnumber=2,
%	numberstyle=\tiny\color{gray},
%	numberblanklines=true,
%	breaklines=true,
%	breakatwhitespace=true,
%	tabsize=2,
%	captionpos=b,
%}

% Monochrome style
\lstdefinestyle{bw}{
	backgroundcolor=\color{white},
	keywordstyle=\bfseries,
	commentstyle=\itshape,
	stringstyle=\slshape,
}

% RAPID language definition
\lstdefinelanguage[]{RAPID}%
{
	keywords={LOCAL,VAR,MODULE,FUNC,ENDFUNC,RETURN,TRUE,FALSE,ENDMODULE,NOT,PROC,ENDPROC,IF,ELSE,ENDIF,THEN,\\PERS, FOR, FROM, TO, DO, ENDFOR, MOD},
	otherkeywords={...},
	comment=[l]{!},%
	morecomment=[s]{/*}{*/},%
	string=[b]{"},%
	alsoletter={\\}
}[keywords,comments,strings]%

% XML language definition
\lstdefinelanguage{XML}
{
	morestring=[s]{"}{"},
	morecomment=[s]{<!--}{-->},
	%stringstyle=\color{black},
	%identifierstyle=\color{blue},
	%keywordstyle=\color{cyan},
	morekeywords={xmlns,version,type}% list your attributes here
}

%% Old style
%% ansys style 
%%\lstloadlanguages{matlab}
%%\lstset{
%%language=ansys,
%%numbers=left, 
%%numberstyle=\scriptsize, 
%%stepnumber=1, 
%%keywordstyle=\color{blue},
%%numbersep=5pt,
%%frame=single,
%%backgroundcolor=\color[rgb]{0.85, 0.85, 0.85},
%%frameround=tttt,
%%basicstyle=\ttfamily\scriptsize,
%%breaklines=true,
%%framextopmargin=50pt,
%%frame=bottomline, 
%%}
%


	\newcommand{\bs}{\boldsymbol}
\newcommand{\mb}{\mathbf}
\newcommand{\vect}[1]{\mathbf{#1}}
\newcommand{\Be}{\begin{equation}}
\newcommand{\Ee}{\end{equation}}

\newcommand{\Bg}{\begin{gather}}
\newcommand{\Eg}{\end{gather}}

\newcommand{\Bi}{\begin{itemize}}
\newcommand{\Ei}{\end{itemize}}

\newcommand{\Bf}{\begin{figure}}
\newcommand{\Ef}{\end{figure}}

\newcommand{\Bt}{\begin{table}}
\newcommand{\Et}{\end{table}}

\newcommand{\Btr}{\begin{tabular}}
\newcommand{\Etr}{\end{tabular}}

\newcommand{\Bc}{\begin{center}}
\newcommand{\Ec}{\end{center}}

\newcommand{\FE}{finite element}
\newcommand{\abs}[1]{\lvert#1\rvert}
\newcommand{\dt}{{t+\Delta t}}
\newcommand{\CR}[1]{\color{black}#1}
\newcommand{\inp}[1]{{\color{blue}{\footnotesize\listinginput{1}{#1}}}}

\newcommand{\ANSYS}{{\sf ANSYS}}
\newcommand{\MATLAB}{{\sf MATLAB}}
\newcommand{\SHELL}[1]{{\sf SHELL#1}}
\newcommand{\SOLID}[1]{{\sf SOLID#1}}
\newcommand{\BEAM}[1]{{\sf BEAM#1}}
\newcommand{\LINK}[1]{{\sf LINK#1}}
\newcommand{\PLANE}[1]{{\sf PLANE#1}}
\newcommand{\SURF}[1]{{\sf SURF#1}}
\newcommand{\BAR}[1]{{\sf BAR#1}}

\newcommand{\ANS}[4]{
{\footnotesize
\noindent#1\bigskip

\noindent#2\bigskip

\noindent {\it Notes}\\
#3\bigskip

\noindent {\it Menu Paths}\\
#4
}
}



% by A. Mehrle
\newcommand{\measure}{\psline[linewidth=0.5pt,arrowsize=4pt]}
\newcommand{\xtick}[1]{\measure(#1,-0.1)(#1,0.1)}
\newcommand{\ytick}[1]{\measure(-0.1,#1)(0.1,#1)}
\newcommand{\link}{\psline[linewidth=2pt]}
\newcommand{\hatch}{\psframe[fillstyle=vlines,linestyle=none,hatchwidth=0.25pt]}
	\AtBeginDocument{\counterwithin{lstlisting}{section}}		%% Zähler ab Section

% Use of Subfiles
	\usepackage{subfiles}


% Counter for continuous roman counting
	\newcounter{romancount}

% Acronyms
	\usepackage[printonlyused]{acronym}
	\newcommand{\listofacronyms}{\newcommand{\listofAkronymsName}{List of Acronyms}		%% Überschrift abhängig von Sprache


\chapter*{\listofAkronymsName}	
%&\hspace{-50mm}
\begin{acronym}[LONGEST] % Longest Acronym here! Macht den Abstand
	\acro{ADC}{Analog-Digital Converter}
	\acro{ARGB}{Alpha, Red, Green, and Blue}	
	
	\acro{BLDC}{Brushless DC}
	
	\acro{CAD}{Computer-Aided Design}
	\acro{COLLADA}{Collaborative Design Activity}
	
	\acro{DFFT}{Discrete Fast Fourier Transform}
	\acro{DoF}{Degree of Freedom}	
	
	\acro{EMF}{Electro Motive Force}
	
	\acro{FFT}{Fast Fourier Transform}	
	\acro{FMI}{Functional Mockup Interface}	
	\acro{FMU}{Functional Mockup Unit}	
	
	\acro{GUI}{Graphical User Interface}
	
	\acro{HIL}{Hardware in the Loop}
	\acro{HMI}{Human-machine interface}
	\acro{HTML}{Hypertext Markup Language}
	\acro{HSV}{Hue, Saturation, and Value}
	
	\acro{IFFT}{Inverse Fast Fourier Transform}
	\acro{IGES}{Initial Graphics Exchange Specification}
	
	\acro{LED}{Light Emitting Diode}
	\acro{MD}{Markdown}
	 
	
	\acro{PDF}{Portable Document Format}
	
	\acro{SIL}{Software in the Loop}
	\acro{STEP}{Standard for the Exchange of Product model data}
	
	\acro{THD}{Total Harmonic Distortion}
	
	\acro{USB}{Universal Serial Bus}
	
	\acro{VCO}{Voltage Controlled Oscillator}
	
	\acro{PWM}{Pulse Width Modulation}

\end{acronym}

\newpage}


% for using \FloatBarrier
% option "section" so that floats to not cross section border by default
	\usepackage[section]{placeins}

% Including pdf pages
	\usepackage[final]{pdfpages}

% LoX as sections (was chapter)
	\makeatletter
	\renewcommand\listoffigures{%
	  \section*{\listfigurename}%
	  \@mkboth{\MakeUppercase\listfigurename}{\MakeUppercase\listfigurename}%
	  \@starttoc{lof}%
	}
	\renewcommand\listoftables{%
	  \section*{\listtablename}%
	  \@mkboth{\MakeUppercase\listtablename}{\MakeUppercase\listtablename}%
	  \@starttoc{lot}%
	}
	\renewcommand\lstlistoflistings{%
	  \section*{\lstlistlistingname}%
	  \@mkboth{\MakeUppercase\lstlistlistingname}{\MakeUppercase\lstlistlistingname}%
	  \@starttoc{lol}%
	}
	\ifthenelse{\boolean{english}}{\renewcommand{\lstlistlistingname}{List of Code}} {\renewcommand{\lstlistlistingname}{Programmcodeverzeichnis}}
	\makeatother

% Blindtext
	\usepackage{blindtext}


% Um Umlaute direkt verwenden zu können 
	\usepackage[utf8]{inputenc}
	\usepackage[T1]{fontenc}
	\usepackage{upgreek}	% Aufrechte Griechische Buchstaben


% Matlab Plots
	\usepackage{tikz}
	\newlength\fwidth
	
	\usepackage{pgfplots} 
	\usepackage{pgfgantt}
	\usepackage{pdflscape}
	\pgfplotsset{compat=newest} 
	\pgfplotsset{plot coordinates/math parser=false}


% Tabellen etc.
	\usepackage{pgfplotstable}
	\usepackage{tabularx}
	\usepackage{booktabs}
	\usepackage{float}
	\usepackage{multirow}
	
	\usepackage{import}

% Chemische Satz
	\usepackage{mhchem}	
