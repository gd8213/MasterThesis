%\selectlanguage{ngerman}
\section*{Abstract}
 In the modern industry automatization is an essential part of many manufacturing companies. When it is used correctly automatization helps saving costs with an increase of quality and quantity in processing products at the same time. Those enormous advantages result in a great demand for well-trained professionals. However, these professionals need qualifications in a wide range of areas like mechanical and electrical engineering, metrology and coding of the control systems. An intense training of theory and its practical realization is needed in order to fulfill the requirements.
The purpose of this bachelor thesis aims at developing a \textit{Teaching Factory} in a laboratory scale so as to improve the training of automatization with a programmable logic controller (\acs{PLC})\acused{PLC}. Relevant learning contents should be prepared clearly and with practical orientation leading to an increased learning success of the users. In addition, this improved knowledge of automatization will result in better chances on the job marked. 
For the purpose of optimizing the benefit of the \textit{Teaching Factory}, the relevant learning content has to be identified in the first step. Next, an overall concept is created from the learning content, which then gets separated into small stations. Finally, a sample solution of the control system and a simulation of a virtual commissioning is created. 
With the finished design, the sample solution and the simulation the \textit{Teaching Factory} represents a platform for teaching automatization. Using this platform will enable future professionals to gain practical experience in handling programmable logic controllers. As a result of splitting the learning content into individual stations, the user will be able to acquire learning goals more flexibly and consequently with fast learning success.


\paragraph*{Keywords:} Programmable logic controll, PLC, education, automatization, university